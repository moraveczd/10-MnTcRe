\documentclass[hyperref=unicode, presentation,10pt]{beamer}

\usepackage[absolute,overlay]{textpos}
\usepackage{array}
\usepackage{graphicx}
\usepackage{adjustbox}
\usepackage{mhchem}
\usepackage{chemfig}
\usepackage[utf8]{inputenc}
\usepackage{caption}

%dělení slov
\usepackage{ragged2e}
\let\raggedright=\RaggedRight
%konec dělení slov

\addtobeamertemplate{frametitle}{
	\let\insertframetitle\insertsectionhead}{}
\addtobeamertemplate{frametitle}{
	\let\insertframesubtitle\insertsubsectionhead}{}

\makeatletter
\CheckCommand*\beamer@checkframetitle{\@ifnextchar\bgroup\beamer@inlineframetitle{}}
\renewcommand*\beamer@checkframetitle{\global\let\beamer@frametitle\relax\@ifnextchar\bgroup\beamer@inlineframetitle{}}
\makeatother
\setbeamercolor{section in toc}{fg=red}
\setbeamertemplate{section in toc shaded}[default][100]

\usepackage{fontspec}
\usepackage{unicode-math}

\usepackage{polyglossia}
\setdefaultlanguage{czech}

\def\uv#1{„#1“}

\mode<presentation>{\usetheme{default}}
 \usecolortheme{crane}

\setbeamertemplate{footline}[frame number]

\title[Crisis]
{C2062 -- Anorganická chemie II}

\subtitle{Mangan, technecium, rhenium a bohrium}
\author{Zdeněk Moravec, hugo@chemi.muni.cz \\ \adjincludegraphics[height=60mm]{img/IUPAC_PSP.jpg}}
\date{}

\begin{document}


\begin{frame}
	\titlepage
\end{frame}

\section{Úvod}
\frame{
	\frametitle{}
	\vfill
	\begin{tabular}{|c|l|l|l|}
	\hline
	 & \textit{Mangan} & \textit{Technecium} & \textit{Rhenium} \\\hline
	 El. konfigurace & 3d$^{5}$ 4s$^{2}$ & 4d$^{5}$ 5s$^{2}$ & 4f$^{14}$ 5d$^{5}$ 6s$^{2}$ \\\hline
	 Teplota tání [$^\circ$C] & 1246 & 2157 & 3186 \\\hline
	 Teplota varu [$^\circ$C]  & 2061 & 4265 & 5630 \\\hline
	 Objeven & 1774 & 1937 & 1908 \\\hline
	 Vzhled & stříbrný\footnote[frame]{Zdroj: \href{https://commons.wikimedia.org/wiki/File:Mangan_1.jpg}{Tomihahndorf/Commons}} & šedý\footnote[frame]{Zdroj: \href{https://commons.wikimedia.org/wiki/File:Technetium-sample.jpg}{MARCO CARDIN/Commons}} & stříbrošedé\footnote[frame]{Zdroj: \href{https://commons.wikimedia.org/wiki/File:Rhenium_single_crystal_bar_and_1cm3_cube.jpg}{Alchemist-hp/Commons}} \\
	 &  \begin{minipage}{.2\textwidth}
	 	\adjincludegraphics[width=\linewidth]{img/Mangan.jpg}
	 \end{minipage}
	 	& \begin{minipage}{.2\textwidth}
	 		\adjincludegraphics[width=\linewidth]{img/Technetium-sample.jpg}
	 	\end{minipage} & \begin{minipage}{.2\textwidth}
	 	\adjincludegraphics[width=\linewidth]{img/Rhenium.jpg}
 	\end{minipage} \\\hline
	\end{tabular}
	\vfill
}

\subsection{Bohrium}
\frame{
	\frametitle{}
	\vfill
	\begin{columns}
		\begin{column}{.6\textwidth}
			\begin{itemize}
				\item Patnáctý transuran, protonové číslo 107, Bh.
				\item Poprvé byl připraven v roce 1976:
				\item \ce{^{209}_{83}Bi + ^{54}_{24}Cr -> ^{261}_{107}Bh$^\star$ -> ^{1}_{0}n}
				\item Bylo pojmenován po dánském fyzikovi Nielsu Bohrovi, který získal Nobelovu cenu za fyziku za výzkum struktury atomu a radioaktivního záření.\footnote[frame]{\href{https://www.nobelprize.org/prizes/physics/1922/summary/}{The Nobel Prize in Physics 1922}}
				\item Název byl schválen roku 1997.\footnote[frame]{\href{https://doi.org/10.1351/pac199769122471}{Names and symbols of transfermium elements (IUPAC Recommendations 1997)}}
			\end{itemize}
		\end{column}
		\begin{column}{.4\textwidth}
			\begin{figure}
				\adjincludegraphics[width=.85\textwidth]{img/Niels_Bohr.jpg}
				\caption*{Niels Bohr.\footnote[frame]{Zdroj: \href{http://nobelprize.org/physics/laureates/1922/bohr-bio.html}{Niels Bohr's Nobel Prize biography}}}
			\end{figure}
		\end{column}
	\end{columns}
	\vfill
}

\frame{
	\frametitle{}
	\vfill
	\begin{tabular}{|l|l|l|}
		\hline
		\textbf{Terč} & \textbf{Projektil} & \textbf{Připravený izotop} \\\hline
		\ce{^{208}Pb} & \ce{^{55}Mn} & \ce{^{263}Bh} \\\hline
		\ce{^{209}Bi} & \ce{^{54}Cr} & \ce{^{263}Bh} \\\hline
		\ce{^{209}Bi} & \ce{^{52}Cr} & \ce{^{261}Bh} \\\hline
		\ce{^{238}U} & \ce{^{31}P} & \ce{^{269}Bh} \\\hline
		\ce{^{243}Am} & \ce{^{26}Mg} & \ce{^{269}Bh} \\\hline
		\ce{^{248}Cm} & \ce{^{23}Na} & \ce{^{271}Bh} \\\hline
		\ce{^{249}Bk} & \ce{^{22}Na} & \ce{^{271}Bh} \\\hline
	\end{tabular}

	\begin{itemize}
		\item Známe 11 radioizotopů a jeden jaderný izomer. Nejstabilnějším izotopem je \ce{^{274}Bh} s poločasem rozpadu 0,9~minut, příp. nepotvrzený izotop \ce{^{275}Bh} s poločasem rozpadu 11,5~minut.
		\item Tento izotop je součástí rozpadové řady flerovia.
		\item V roce 2000 byla publikována příprava \ce{BhO3Cl}:\footnote[frame]{\href{https://www.superheavies.de/english/publications/pub_images/pub_annual_img/annual_reports_2000/SHE_AR_2000_eichler.pdf}{Gas chemical investigation of bohrium}}
		\item \ce{2 Bh + 3 O2 + 2 HCl ->[1000 $^\circ$C] 2 BhO3Cl + H2}
	\end{itemize}
	\vfill
}

\section{Chemické a fyzikální vlastnosti}
\frame{
	\frametitle{}
	\vfill
	\begin{itemize}
		\item Mangan a rhenium jsou stabilní prvky, technecium je nejlehčí prvek, který nemá stabilní izotop.
		\item Manganu se ročně spotřebují milióny tun, naproti tomu technecium i~rhenium jsou vzácné prvky.
		\item Jejich elektronová konfigurace je (n-1)d$^5$ ns$^2$.
		\item Mají vysoké teploty tání.
		\item Mangan vytváří čtyři allotropy, za laboratorní teploty je stabilní $\alpha$ modifikace krystalující v kubické prostorově centrované buňce.
		\item Technecium a rhenium krystalují v nejtěsnějším hexagonálním uspořádání.
		\item Mangan je tvrdý a křehký.
		\item Rhenium je těžko tavitelné, má druhou nejvyšší teplotu tání mezi přechodnými kovy.
		\item Nejstabilnější oxidační číslo manganu je II, nejvyšší hodnota oxidačního čísla je VII, tyto sloučeniny mají silnější oxidační vlastnosti než sloučeniny \ce{Cr^{VI}}.
	\end{itemize}
	\vfill
}

\section{Výskyt a získávání prvků}
\subsection{Mangan}
\frame{
	\frametitle{}
	\vfill
	\begin{itemize}
		\item Koncentrace manganu v zemské kůře je asi 0,1~\%, jde o 12. nejzastoupenější prvek.
		\item Nejdůležitějším minerálem je pyrolusit, \ce{MnO2}.
		\item Další významné minerály jsou braunit, psilomelan a rhodochrosit.
		\item Známe téměř 500 minerálů obsahujících mangan.\footnote[frame]{\href{https://www.mindat.org/element/Manganese}{The mineralogy of Manganese}}
		\item Mangan je také přítomen ve sfaleritech (\ce{(Zn,Fe)S}) s vysokým obsahem železa.\footnote[frame]{\href{https://www.mindat.org/min-3727.html}{Sphalerite}}
		\item Mangan se těží převážně v Jižní Africe, Číně, Austrálii, Brazílii a~Indii.
	\end{itemize}
	\begin{figure}
		\adjincludegraphics[height=0.2\textwidth]{img/World_Manganese_Production_2006.png}
		\caption*{Světová produkce manganu v roce 2007.\footnote[frame]{Zdroj: \href{https://commons.wikimedia.org/wiki/File:World_Manganese_Production_2006.svg}{Stone/Commons}}}
	\end{figure}
	\vfill
}

\frame{
	\frametitle{}
	\vfill
	\textbf{Pyroluzit}
	\begin{itemize}
		\item Tetragonální minerál, \ce{MnO2}, černá až šedá barva.\footnote[frame]{\href{https://mineraly.sci.muni.cz/oxidy/pyroluzit.html}{Pyroluzit}}
		\item Má strukturu rutilu.\footnote[frame]{\href{https://www.mindat.org/min-3318.html}{Pyrolusite}}
		\item Hlavní ruda manganu.
	\end{itemize}
	\begin{columns}
		\begin{column}{.5\textwidth}
			\begin{figure}
				\adjincludegraphics[height=.33\textheight]{img/Pyrolusite_botryoidal.jpg}
				\caption*{Pyroluzit, Brazílie.\footnote[frame]{Zdroj: \href{https://commons.wikimedia.org/wiki/File:Pyrolusite_botryoidal.jpg}{Aram Dulyan/Commons}}}
			\end{figure}
		\end{column}
		\begin{column}{.5\textwidth}
			\begin{figure}
				\adjincludegraphics[height=.33\textheight]{img/Piroluzyt_Wlochy.jpg}
				\caption*{Pyroluzit, Itálie.\footnote[frame]{Zdroj: \href{https://commons.wikimedia.org/wiki/File:Piroluzyt,_W\%C5\%82ochy.jpg}{Kluka/Commons}}}
			\end{figure}
		\end{column}
	\end{columns}
	\vfill
}

\frame{
	\frametitle{}
	\vfill
	\textbf{Braunit}
	\begin{itemize}
		\item Tetragonální minerál, \ce{Mn^{2+}Mn^{3+}6[O8|SiO4]}, hnědá až šedá barva.\footnote[frame]{\href{https://www.mindat.org/min-757.html}{Braunit}}
		\item Nesosilikát.
	\end{itemize}
	\begin{columns}
		\begin{column}{.5\textwidth}
			\begin{figure}
				\adjincludegraphics[height=.4\textheight]{img/Braunite-229837.jpg}
				\caption*{Braunit, Afrika.\footnote[frame]{Zdroj: \href{https://commons.wikimedia.org/wiki/File:Braunite-229837.jpg}{Robert M. Lavinsky/Commons}}}
			\end{figure}
		\end{column}
		\begin{column}{.5\textwidth}
			\begin{figure}
				\adjincludegraphics[height=.4\textheight]{img/Braunite_Manganese_oxide_and_silicate.jpg}
				\caption*{Braunit, Japonsko.\footnote[frame]{Zdroj: \href{https://commons.wikimedia.org/wiki/File:Braunite_Manganese_oxide_and_silicate_Kuratomi_Mine,_Tsukumi-shi,_Oita-ken,_Kyushu,_Japan_2382.jpg}{Dave Dyet/Commons}}}
			\end{figure}
		\end{column}
	\end{columns}
	\vfill
}

\frame{
	\frametitle{}
	\vfill
	\textbf{Psilomelan}
	\begin{itemize}
		\item Monoklinické minerály, \ce{Ba(Mn^{2+})(Mn^{4+})8O16(OH)4} nebo\\ \ce{(Ba,H2O)2Mn5O10}, černá až šedá barva.\footnote[frame]{\href{https://mineraly.sci.muni.cz/oxidy/psilomelan.html}{Psilomelan}}
		\item Skupina oxidických minerálů.\footnote[frame]{\href{https://www.mindat.org/min-3304.html}{Psilomelane}}
		\item Složení je proměnlivé.
	\end{itemize}
	\begin{columns}
		\begin{column}{.5\textwidth}
			\begin{figure}
				\adjincludegraphics[height=.3\textheight]{img/Psilomelane-167850.jpg}
				\caption*{Psilomelan, USA.\footnote[frame]{Zdroj: \href{https://commons.wikimedia.org/wiki/File:Psilomelane-167850.jpg}{Robert M. Lavinsky/Commons}}}
			\end{figure}
		\end{column}
		\begin{column}{.5\textwidth}
			\begin{figure}
				\adjincludegraphics[height=.3\textheight]{img/Vanadinite-Psilomelane-112809.jpg}
				\caption*{Psilomelan a vanadinit, Maroko.\footnote[frame]{Zdroj: \href{https://commons.wikimedia.org/wiki/File:Vanadinite-Psilomelane-112809.jpg}{Robert M. Lavinsky/Commons}}}
			\end{figure}
		\end{column}
	\end{columns}
	\vfill
}

\frame{
	\frametitle{}
	\vfill
	\textbf{Rodochrozit}
	\begin{itemize}
		\item Trigonální minerál, \ce{MnCO3}, růžová, červená až žlutá barva.\footnote[frame]{\href{https://mineraly.sci.muni.cz/karbonaty/rodochrozit.html}{Rodochrozit}}
		\item Biominerál, může být produkován houbami během oxidačních procesů.\footnote[frame]{\href{https://www.mindat.org/min-3406.html}{Rhodochrosite}}
		\item Využívá se k výrobě manganových slitin.
	\end{itemize}
	\begin{columns}
		\begin{column}{.5\textwidth}
			\begin{figure}
				\adjincludegraphics[height=.3\textheight]{img/Rhodochrosite_Crystal.jpg}
				\caption*{Rodochrozit, Colorado.\footnote[frame]{Zdroj: \href{https://commons.wikimedia.org/wiki/File:The_Searchlight_Rhodochrosite_Crystal.jpg}{Eric Hunt/Commons}}}
			\end{figure}
		\end{column}
		\begin{column}{.5\textwidth}
			\begin{figure}
				\adjincludegraphics[height=.3\textheight]{img/Rhodochrosite_Pink_Form.jpg}
				\caption*{Růžová modofikace rodochrozitu, Colorado.\footnote[frame]{Zdroj: \href{https://commons.wikimedia.org/wiki/File:Rhodochrosite_Pink_Form.jpg}{Eric Hunt/Commons}}}
			\end{figure}
		\end{column}
	\end{columns}
	\vfill
}

\frame{
	\frametitle{}
	\vfill
	\begin{itemize}
		\item Část manganových rud se zpracovává na \textit{ferromangan}.
		\item Manganová ruda se redukuje s železnou rudou a koksem ve vysokých nebo elektrických pecích.\footnote[frame]{\href{https://a.storyblok.com/f/94542/x/542c5a5c32/ferromanganese-data-sheet.pdf}{Ferromanganese}}
		\item Často se redukce provádí v přítomnosti vápence, který váže křemík a vytváří strusku.
		\item Obsahuje 75~\% manganu a 7~\% uhlíku.
		\item Světová produkce se pohybuje v miliónech tun.
	\end{itemize}
	\begin{figure}
		\adjincludegraphics[height=0.3\textwidth]{img/Ferromanganese.jpg}
		\caption*{Ferromangan.\footnote[frame]{Zdroj: \href{https://commons.wikimedia.org/wiki/File:Ferromangan\%C3\%A8se_m\%C3\%A9tal.jpg}{Borvan53/Commons}}}
	\end{figure}
	\vfill
}

\frame{
	\frametitle{}
	\vfill
	\begin{itemize}
		\item Surový mangan se získává redukcí železem.
		\item Ruda se redukuje zemním plynem, který vystupuje jako zdroj tepla i~redukčního činidla (CO).
		\item Redukcí získáme burel, \ce{MnO2}. Mletím je snížen průměr částic na 150--250~\ce{$\mu$}m, tím dojde ke zvýšení měrného povrchu a usnadnění extrakce.
		\item Extrakce se provádí kyselinou sírovou s rozpuštěnou železnatou solí.
		\item Železnatá sůl redukuje manganičité ionty na kovový mangan.
		\item Takto se získá více než 90~\% manganu.
		\item Další čištění je možné provést elektrolyticky.\footnote[frame]{\href{https://doi.org/10.1016/j.hydromet.2016.01.010}{Electrolytic manganese metal production from manganese carbonate precipitate}}
	\end{itemize}
	\vfill
}

\frame{
	\frametitle{}
	\vfill
	\begin{itemize}
		\item Standardní elektrodový potenciál \ce{Mn^{2+}/Mn} je $-$1,18~V.
		\item Ruda se zpracuje na oxid manganatý.\footnote[frame]{\href{https://www.saimm.co.za/Journal/v077n07p137.pdf}{The production of electrolytic manganese in South Africa}}
		\item Oxid manganatý se rozpustí v kyselině sírové:
		\item \ce{MnO + H2SO4 -> MnSO4 + H2O}
		\item Nečistoty jsou z roztoku odstraněny srážením sulfidem amonným.
		\item Čistý roztok je poté elektrolyzován:
		\item Katoda: \ce{MnSO4 + 2 e- -> Mn + SO$_4^{2-}$}
		\item Anoda: \ce{H2O -> 2 H+ + 1/2 O2 + 2 e-}
		\item Celková reakce: \ce{MnSO4 + H2O -> Mn + H2SO4 + 1/2 O2}
	\end{itemize}
	\vfill
}

\subsection{Technecium}
\frame{
	\frametitle{}
	\vfill
	\begin{itemize}
		\item Technecium se vyskytuje pouze ve stopových množstvích.
		\item V přírodě se vyskytuje izotop \ce{^{99}Tc} s poločasem rozpadu 2,12.10$^5$ roků.
		\item Uměle se připravuje štěpením \ce{^{238}U} pomalými neutrony.
		\item \ce{$^{238}_{\ 92}$U -> $^{137}_{\ 53}$I + $^{99}_{39}$Y + 2 ^1_0n}
		\item \ce{$^{99}_{39}$Y ->[$\beta$^-][1.47 s] $^{99}_{40}$Zr ->[$\beta$^-][2.1 s] $^{99}_{41}$Nb ->[$\beta$^-][15.0 s] $^{99}_{42}$Mo ->[$\beta$^-][65.94 h] $^{99}_{43}$Tc}
		\item Tvoří až 6~\% jaderného odpadu.
		\item Z něj se získává až po několika letech, aby došlo k rozpadu jader s kratším poločasem rozpadu a tím snížení aktivity odpadu.
		\item Roztok po izolaci plutonia a uranu v procesu PUREX obsahuje podíl technecistanu, \ce{^{99}TcO$_4^-$}.
		\item Ten lze extrahovat pyridem a poté vykrystalovat jako technecistan amonný, \ce{NH4$^{99}$TcO4}.
	\end{itemize}
	\vfill
}

\frame{
	\frametitle{}
	\vfill
	\begin{itemize}
		\item Nejpoužívanější je jaderný izomer \ce{^{99m}Tc}, který se získává z generátoru \ce{^{99}Mo/^{99m}Tc}.
		\item Mateřský izotop \ce{^{99}Mo} je ve formě \ce{MoO$^{2-}_4$} immobilizován v horní části anexové kolony naplněné aluminou.
		\item Vznikající \ce{^{99m}TcO$^{-}_4$} je eluován roztokem NaCl.
		\item Izomer přechází do základního stavu emisí fotonu o energii 140~keV, poločas 6,0 hodin.
		\item V malé míře dochází také k rozpadu $\beta^-$:
		\item \ce{^{99m}Tc -> ^{99}Ru + $\beta$^-}
	\end{itemize}

	\begin{figure}
		\adjincludegraphics[height=0.25\textwidth]{img/Five99mTechnetiumGenerators.jpg}
		\caption*{Generátory \ce{^{99}Mo/^{99m}Tc}.\footnote[frame]{\href{https://commons.wikimedia.org/wiki/File:Five99mTechnetiumGenerators.jpg}{Zdroj: Kieran Maher/Commons}}}
	\end{figure}
	\vfill
}

\subsection{Rhenium}
\frame{
	\frametitle{}
	\vfill
	\begin{itemize}
		\item Rhenium je jedním z nejvzácnějších prvků, jeho koncentrace v zemské kůře je okolo 1 ppb.
		\item Vyskytuje se ve dvou minerálech: \textit{rheniitu} (\ce{ReS2}) a \textit{tarkianitu}\\ (\ce{(Cu,Fe)(Re,Mo)4S8}).\footnote[frame]{\href{https://www.mindat.org/element/Rhenium}{The mineralogy of Rhenium}}
		\item Komerčním zdrojem rhenia je molybdenit, který obsahuje asi 0,2~\% rhenia.
		\item Největší zdroje rhenia se nacházejí v Chile.
	\end{itemize}
	\begin{figure}
		\adjincludegraphics[height=0.3\textwidth]{img/Rhenium.jpg}
		\caption*{Rhenium.\footnote[frame]{Zdroj: \href{https://commons.wikimedia.org/wiki/File:Rhenium_single_crystal_bar_and_1cm3_cube.jpg}{Alchemist-hp/Commons}}}
	\end{figure}
	\vfill
}

\frame{
	\frametitle{}
	\vfill
	\textbf{Molybdenit}
	\begin{itemize}
		\item Hexagonální minerál, \ce{MoS2}, modravě šedá barva.\footnote[frame]{\href{https://mineraly.sci.muni.cz/sulfidy/molybdenit.html}{Molybdenit}}
		\item Využívá se v ocelářském a chemickém průmyslu.
		\item Dříve se krystaly molybdenitu (nebo \ce{FeS2}, \ce{PbCO3}) využívaly ke konstrukci hrotových diod (cat's whisker detectors) určených k demodulaci rádiového signálu.
	\end{itemize}
	\begin{columns}
		\begin{column}{.5\textwidth}
			\begin{figure}
				\adjincludegraphics[height=.33\textheight]{img/Molybdenite.jpg}
				\caption*{Molybdenit, Mexiko.\footnote[frame]{Zdroj: \href{https://commons.wikimedia.org/wiki/File:Molybdenite_(Questa,_New_Mexico)_2_(19057184939).jpg}{James St. John/Commons}}}
			\end{figure}
		\end{column}
		\begin{column}{.5\textwidth}
			\begin{figure}
				\adjincludegraphics[height=.33\textheight]{img/Molybdenite_quebec2.jpg}
				\caption*{Molybdenit na křemeni, Kanada.\footnote[frame]{Zdroj: \href{https://commons.wikimedia.org/wiki/File:Molybdenite_quebec2.jpg}{Didier Descouens/Commons}}}
			\end{figure}
		\end{column}
	\end{columns}
	\vfill
}

\frame{
	\frametitle{}
	\vfill
	\textbf{Rheniit}
	\begin{itemize}
		\item Trojklonný minerál, \ce{ReS2}, černá až stříbrno-bílá barva.\footnote[frame]{\href{https://www.mindat.org/min-7269.html}{Rheniite}}
		\item Velmi vzácný, poprvé byl nalezen v roce 1994 v Rusku.\footnote[frame]{\href{https://doi.org/10.1038/369051a0}{Discovery of a pure rhenium mineral at Kudriavy volcano}}
	\end{itemize}
	\begin{columns}
		\begin{column}{.5\textwidth}
			\begin{figure}
				\adjincludegraphics[height=.4\textheight]{img/Rheniite-183985.jpg}
				\caption*{Rheniit, Rusko.\footnote[frame]{Zdroj: \href{https://commons.wikimedia.org/wiki/File:Rheniite-183985.jpg}{Robert M. Lavinsky/Commons}}}
			\end{figure}
		\end{column}
		\begin{column}{.5\textwidth}
			\begin{figure}
				\adjincludegraphics[height=.4\textheight]{img/Rheniite-34295.jpg}
				\caption*{Rheniit na lávě, Rusko.\footnote[frame]{Zdroj: \href{https://commons.wikimedia.org/wiki/File:Rheniite-34295.jpg}{Robert M. Lavinsky/Commons}}}
			\end{figure}
		\end{column}
	\end{columns}
	\vfill
}

\frame{
	\frametitle{}
	\vfill
	\begin{itemize}
		\item Rhenium se vyrábí z polétavých prachů vznikajících pražením molybdenitu.
		\item Kov je oxidován na \ce{Re2O7} a oxid je poté srážen chloridem amonným:
		\item \ce{Re2O7 + 2 NH4Cl + H2O -> 2 NH4ReO4 + 2 HCl}
		\item Kovové rhenium se připravuje redukcí rhenistanu amonného vodíkem:\footnote[frame]{\href{https://doi.org/10.1080/03719553.2017.1375707}{Investigation on ammonium perrhenate behaviour in nitrogen, argon and hydrogen atmosphere as a part of rhenium extraction process}}
		\item \ce{2 NH4ReO4 + 7 H2 ->[300 $^\circ$C] 2 Re + 8 H2O + 2 NH3}
		\item Celosvětová roční produkce rhenia se pohybuje okolo 50 tun.\footnote[frame]{\href{https://www.usgs.gov/centers/nmic/rhenium-statistics-and-information}{Rhenium Statistics and Information}}
	\end{itemize}
	\vfill
}

\section{Využití prvků}
\subsection{Mangan}
\frame{
	\frametitle{}
	\begin{columns}
		\begin{column}{.7\textwidth}
			\vfill
			\begin{itemize}
				\item Většina manganu se využívá ve slitinách, nejčastěji v železných (v ocelích), ale i~v~hliníkových.
				\item V ocelářství se využívá \textit{ferromangan}, což je slitina manganu a železa s obsahem až 80~\% manganu. Vyrábí se redukcí směsi oxidu manganičitého a železitého koksem ve vysoké nebo elektrické peci.\footnote[frame]{\href{https://www.britannica.com/technology/manganese-processing}{Manganese processing}}
				\item Mangan je při výrobě ocelí nenahraditelný, slouží k fixaci síry. Zabraňuje vzniku sulfidů železa na hranicích zrn.
				\item Váže rozpuštěný kyslík, síru a fosfor.
				\item Malá množství manganu zvyšují opracovatelnost ocelí za vyšší teploty, vytváří totiž sulfidy, které mají vysokou teplotu tání.
			\end{itemize}
			\vfill
		\end{column}
		\begin{column}{.35\textwidth}
			\begin{figure}
				\adjincludegraphics[width=\textwidth]{img/Ferromanganese_metal.jpg}
				\caption*{Ferromangan.\footnote[frame]{Zdroj: \href{https://commons.wikimedia.org/wiki/File:Ferromangan\%C3\%A8se_m\%C3\%A9tal.jpg}{Borvan53/Commons}}}
			\end{figure}
		\end{column}
	\end{columns}
}

\frame{
	\frametitle{}
	\vfill
	\begin{itemize}
		\item Ve slitinách s hliníkem se používá jen malý podíl manganu, do 1,5~\%. Tyto slitiny mají vyšší pevnost a lepší odolnost vůči korozi než čistý hliník.
	\end{itemize}
	\begin{tabular}{|l|l|l|}
		\hline
		\textbf{Slitina} & \textbf{Obsah Al [\%]} & \textbf{Obsah legur} \\\hline
		3003 & 98,4 & Mn 1,5; Cu 0,12 \\\hline
		3004 & 97,8 & Mn 1,2; Mg 1 \\\hline
		3005 & 98,5 & Mn 1,0; Mg 0,5 \\\hline
		3102 & 99,8 & Mn 0,2 \\\hline
		3103 & 98,8 & Mn 1,2 \\\hline
		3105 & 97,8 & Mn 0,55; Mg 0,5 \\\hline
		3203 & 98,8 & Mn 1,2 \\\hline
		3303 & 98,8 & Mn 1,2 \\\hline
		4015 & 96,8 & Si 2,0; Mn 1,0; Mg 0,2 \\\hline
		5026 & 93,9 & Mg 4,5; Mn 1; Si 0,9; Fe 0,4; Cu 0,3 \\\hline
	\end{tabular}
	\vfill
}

\frame{
	\frametitle{}
	\vfill
	\begin{itemize}
		\item Trikarbonyl(methylcyklopentadienyl)mangan, MMT, se používá jako náhrada tetraethylolova.\footnote[frame]{\href{https://echa.europa.eu/cs/substance-information/-/substanceinfo/100.031.957}{Tricarbonyl(methylcyclopentadienyl)manganese}}
		\item Zvyšuje oktanové číslo benzínu.
		\item Připravuje se redukcí bis(methylcyklopentadienyl)manganatého komplexu triethylhliníkem v atmosféře CO.
		\item Reakce je silně exotermní, bez chlazení může vést k výbuchu.\footnote[frame]{\href{https://www.youtube.com/watch?v=C561PCq5E1g}{Runaway: Explosion at T2 Laboratories}}
	\end{itemize}
	\begin{figure}
		\adjincludegraphics[height=0.3\textwidth]{img/Methylcyclopentadienyl-Manganese-Tricarbonyl_Skeletal.png}
	\end{figure}
	\vfill
}

\frame{
	\frametitle{}
	\vfill
	\begin{itemize}
		\item \textit{Oxid manganičitý}, \textit{burel}, \ce{MnO2}.
		\item Využívá se jako pigment v keramice a sklářství, v suchých článcích a~organické syntéze.
		\item V dnešní době jsou běžnější alkalické články, kde je anoda tvořena práškovým zinkem v hydroxidu draselném, starší zinko-uhlíkové články mají anodu tvořenou zinkovým pláštěm článku.
		\item V suchých článcích vystupuje, ve směsi s mletým grafitem, jako katoda.
		\item \ce{Zn + 2 Cl- -> ZnCl2 + 2 e-}
		\item \ce{2 MnO2 + 2 NH4Cl + H2O + 2 e- -> Mn2O3 + 2 NH4OH + 2 Cl-}
	\end{itemize}
	\begin{figure}
		\adjincludegraphics[height=0.25\textheight]{img/Zincbattery.png}
		\caption*{Schéma a řez zinko-uhlíkovým článkem.\footnote[frame]{Zdroj: \href{https://commons.wikimedia.org/wiki/File:Zincbattery_(1).png}{Mcy jerry/Commons}}}
	\end{figure}
	\vfill
}

\frame{
	\frametitle{}
	\vfill
	\begin{itemize}
		\item V organické syntéze se využívá \ce{MnO2} jako oxidační činidlo.
		\item Oxiduje alkoholy na karbonyly:\footnote[frame]{\href{https://dx.doi.org/10.1002/047084289X.rm021.pub4}{Encyclopedia of Reagents for Organic Synthesis || Manganese Dioxide.}}
		\item \ce{RCH=CHCH2OH + MnO2 -> RCH=CHCHO + MnO + H2O}
	\end{itemize}

	\begin{figure}
		\adjincludegraphics[width=\textwidth]{img/MnO2-catal.png}
	\end{figure}
	\vfill
}

\frame{
	\frametitle{}
	\vfill
	\begin{itemize}
		\item \textit{Manganistan draselný}, \ce{KMnO4}.
		\item Ročně se ho vyrobí asi 500 000 tun.
		\item Je to velice účinné oxidační činidlo.
		\item Využívá se i v lékařství jako dezinfekce.
	\end{itemize}
	\begin{columns}
		\begin{column}{.7\textwidth}
			\begin{figure}
				\adjincludegraphics[height=0.4\textheight]{img/KMnO4_and_glycerol.jpg}
				\caption*{Reakce manganistanu s glycerolem.\footnote[frame]{Zdroj: \href{https://commons.wikimedia.org/wiki/File:KMnO4_and_glycerol.jpg}{Adam Redzikowski/Commons}}}
			\end{figure}
		\end{column}
		\begin{column}{.3\textwidth}
			\begin{figure}
				\adjincludegraphics[height=0.4\textheight]{img/KMN04-4.jpg}
				\caption*{Rozpouštění \ce{KMnO4} ve vodě.\footnote[frame]{Zdroj: \href{https://commons.wikimedia.org/wiki/File:Difuzija_KMN04-4.jpg}{Asistent ISP/Commons}}}
			\end{figure}
		\end{column}
	\end{columns}
	\vfill
}

\frame{
	\frametitle{}
	\vfill
	\begin{itemize}
		\item Hlavní využití nachází manganistan v organické syntéze.\footnote[frame]{\href{https://dx.doi.org/10.1055/s-1987-27859}{The Classical Permanganate Ion: Still a Novel Oxidant in Organic Chemistry}}
		\item Alkeny dokáže oxidovat na dieny.\footnote[frame]{\href{https://chem.libretexts.org/Bookshelves/Organic_Chemistry/Supplemental_Modules_(Organic_Chemistry)/Reactions/Oxidation_and_Reduction_Reactions/Oxidation_of_Organic_Molecules_by_KMnO4}{Oxidation of Organic Molecules by \ce{KMnO4}}}
		\item Alkyny oxiduje na diony, terminální alkyny na karboxylové kyseliny.
		\item Alkoholy oxiduje na karbonyly.
	\end{itemize}
	\begin{figure}
		\adjincludegraphics[width=\textwidth]{img/Baeyers_Probe_Alkenes_V.png}
	\end{figure}
	\vfill
-}

\frame{
	\frametitle{}
	\vfill
	\textbf{Manganometrie}
	\begin{itemize}
		\item \textit{Manganometrie} je metoda redoxní odměrné analýzy.
		\item Odměrným roztokem je manganistan draselný.
		\item Titrace se provádí v kyselém prostředí.
		\item \ce{MnO$_4^-$ + 8 H+ + 5 e- -> Mn^{2+} + 4 H2O}
		\item Jako indikátor se využívají první kapky nezreagovaného manganistanu.
		\item Manganometrií je možné stanovit koncentraci železa, dusitanů, peroxidů i organických analytů:
		\item \ce{5 CH3OH + 6 MnO$_4^-$ + 18 H+ -> 5 CO2 + 6 Mn^{2+} + 19 H2O}
		\item Standardizace se provádí na roztok kyseliny šťavelové nebo šťavelanu sodného.
		\item \small \ce{2 KMnO4 + 6 HCl + 5 (COOH)2 -> 2 MnCl2 + 10 CO2 + 2 KCl + 8 H2O}
	\end{itemize}
	\vfill
}

\subsection{Technecium}
\frame{
	\frametitle{}
	\vfill
	\begin{itemize}
		\item Jaderný izomer \ce{^{99m}Tc} se využívá v medicíně k zobrazování orgánů.
		\item Radiofarmaka obsahující izotop \ce{^{99m}Tc} se využívají ke studiu mozku, srdce, jater ledvin a dalších orgánů.
		\item Izomer \ce{^{95m}Tc} s poločasem rozpadu 61 dnů se používá ke studiu pohybu technecia v životním prostředí a v živočišných i rostlinných organismech.
	\end{itemize}
	\begin{figure}
		\adjincludegraphics[width=.8\textwidth]{img/99Tc_bone_scintigraphy_showed_bull’s_head_sign.jpg}
		\caption*{$^{99}$Tc scintigrafie skeletu.\footnote[frame]{Zdroj: \href{https://commons.wikimedia.org/wiki/File:99Tc_bone_scintigraphy_showed_bull\%E2\%80\%99s_head_sign.jpg}{Koichiro Yamamoto, Hiroyuki Honda, Hideharu Hagiya, Fumio Otsuka/Commons}}}
	\end{figure}
	\vfill
}

\subsection{Rhenium}
\frame{
	\frametitle{}
	\vfill
	\begin{itemize}
		\item Hlavní využití rhenia je ve slitinách odolných vůči vysokým teplotám. Část se také využívá v katalýze.\footnote[frame]{\href{https://doi.org/10.3103/S1067821207060089}{Rhythms of rhenium}}
		\item Slitiny rhenia se využívají ve vysokoteplotních aplikacích, např. v~proudových motorech
	\end{itemize}
	\begin{figure}
		\adjincludegraphics[height=0.4\textwidth]{img/Engine.f15.arp.750pix.jpg}
		\caption*{Test proudového motoru.\footnote[frame]{Zdroj: \href{https://commons.wikimedia.org/wiki/File:Engine.f15.arp.750pix.jpg}{U.S. Air Force/Commons}}}
\end{figure}
	\vfill
}

\frame{
	\frametitle{}
	\vfill
	\begin{itemize}
		\item Rhenium zvyšuje opracovatelnost wolframu za nízkých teplot.
		\item Také zlepšuje stabilitu slitin za vysokých teplot.
		\item Z toho důvodu se vyrábějí slitiny wolframu s rheniem až do koncentrace 27~\% rhenia, což je jeho maximální rozpustnost.
		\item Slitina wolframu s rheniem se také využívá jako zdroj RTG záření.
		\item Vyrábí se z ní vlákna používaná v MS spektrometrii a manometrech.
	\end{itemize}
	\begin{figure}
		\adjincludegraphics[height=0.4\textheight]{img/Illustration_of_Crookes_X-ray_tube.png}
		\caption*{Schéma zdroje RTG záření.\footnote[frame]{Zdroj: \href{https://commons.wikimedia.org/wiki/File:Illustration_of_Crookes_X-ray_tube.svg}{Jhelebrant/Commons}}}
	\end{figure}
	\vfill
}

\section{Sloučeniny}
\frame{
	\frametitle{}
	\vfill
	\begin{itemize}
		\item Mangan je poměrně reaktivní, na vzduchu se oxiduje, v jemném stavu je pyroforický.
		\item Nejstabilnější oxidační číslo je II.
		\item Sloučeniny \ce{Mn^{VII}} jsou silná oxidovadla.
		\item Maximální koordinační číslo je 6.
		\item Rozkládá vodu za vzniku vodíku.
		\item Ve zředěných kyselinách se rozpouští za vzniku manganatých solí.
		\item S nekovy reaguje až za vyšších teplot.
		\item V kyslíku, dusíku, chloru a fluoru hoří:
	\end{itemize}
	\begin{align*}
		\ce{3 Mn + 2 O2 &-> Mn3O4} \\
		\ce{3 Mn + N2 &-> Mn3N2} \\
		\ce{Mn + Cl2 &-> MnCl2} \\
		\ce{4 Mn + 5 F2 &-> 2 MnF2 + 2 MnF3}
	\end{align*}
	\vfill
}

\frame{
	\frametitle{}
	\vfill
	\begin{itemize}
		\item Technecium i rhenium jsou méně reaktivní a jsou si chováním velmi podobné.
		\item Vytvářejí sloučeniny v oxidačních číslech 0 až VII.
		\item Sloučeniny v oxidačním čísle II jsou vzácné.
		\item Sloučeniny \ce{M^{VII}} mají jen slabé oxidační vlastnosti.
		\item Informací o chemii technecia není příliš, především kvůli jeho radioaktivitě.
		\item Rhenium dosahuje ve sloučeninách až koordinačního čísla 9.
		\item Zahříváním v kyslíku hoří na \ce{M2O7}.
		\item Se sírou vzniká \ce{MS2}.
		\item S fluorem vznikají směsi \ce{TcF5 + TcF6} a \ce{ReF6 + ReF7}.
	\end{itemize}
	\vfill
}

\frame{
	\frametitle{}
	\vfill
	\begin{tabular}{|l|l|l|}
		\hline
		\textbf{Oxidační číslo} & \textbf{Mn} & \textbf{Tc/Re} \\\hline
		$-$III & \ce{[Mn(CO)(NO)3]} & \ce{[M(CO)4]^{3-}} \\\hline
		$-$II & \ce{[Mn(ftalocyanin)]^{2-}} & - \\\hline
		$-$I & \ce{[Mn(CO)5]^-} & \ce{[M(CO)5]^{-}} \\\hline
		0 & \ce{[Mn2(CO)10]} & \ce{[M2(CO)10]} \\\hline
		I & \ce{[Mn2(CO)10]} & \ce{[M2(CO)10]} \\\hline
		II & \ce{[MnBr4]^{2-}} & \ce{[M2Cl2(diars)2]} \\\hline
		III & \ce{[Mn(bipy)(NO3)3]} & \ce{[M(CN)7]^{4-}} \\\hline
		IV & \ce{[MnF6]^{2-}} & \ce{[MI6]^{2-}} \\\hline
		V & \ce{[MnO4]^{3-}} & \ce{[MOCl4]^{-}} \\\hline
		VI & \ce{[MnO4]^{2-}} & \ce{[ReOCl4]} \\\hline
		VII & \ce{[MnO4]^{-}} & \ce{[MO4]^{-}} \\\hline
	\end{tabular}

	* Ligand \ce{NO+} -- nitrosonium\\
	* diars -- 1,2-Bis(dimethylarsino)benzene
	\vfill
}

\subsection{Sloučeniny manganu v oxidačním stavu VII}
\frame{
	\frametitle{}
	\vfill
	\textit{Oxidační stav VII}
	\begin{itemize}
		\item Nejvýznamnější sloučeninou je manganistan draselný, \ce{KMnO4}.
		\item Koncentrovaná kyselina sírová ho převádí na oxid:
		\item \ce{2 KMnO4 + H2SO4 -> Mn2O7 + K2SO4 + H2O}
		\item Oxid manganistý se i za nízkých teplot rozkládá na \ce{MnO2}, někdy i~explozivně:
		\item \ce{2 Mn2O7 ->[$-10^\circ$C] 4 MnO2 + 3 O2}
		\item Kyselinu manganistou, \ce{HMnO4}, lze získat odpařením roztoku (získaného pomocí iontoměničů) za nízké teploty.
		\item Binární halogenidy nebyly dosud izolovány.
		\item Reakcí manganistanu s halogenidem kyseliny sírové získáme oxid-halogenidy:
		\item \ce{KMnO4 + 2 HSO3F -> MnO3F + KSO3F + H2SO4}
		\item \ce{KMnO4 + 2 HSO3Cl -> MnO3Cl + KSO3Cl + H2SO4}
		\item Obě tyto sloučeniny jsou silná oxidační činidla.
	\end{itemize}
	\vfill
}

\frame{
	\frametitle{}
	\vfill
	\textbf{Manganistan draselný}
	\begin{columns}
		\begin{column}{.5\textwidth}
			\begin{figure}
				\adjincludegraphics[width=\textwidth]{img/KMnO4.jpg}
				\caption*{Manganistan draselný.\footnote[frame]{Zdroj: \href{https://commons.wikimedia.org/wiki/File:Manganistan_draseln\%C3\%BD.JPG}{Ondřej Mangl/Commons}}}
			\end{figure}
		\end{column}
		\begin{column}{.5\textwidth}
			\begin{figure}
				\adjincludegraphics[width=\textwidth]{img/KMnO4-struct.png}
			\end{figure}
		\end{column}
	\end{columns}
	\vfill
}

\subsection{Sloučeniny manganu v oxidačním stavu VI}
\frame{
	\frametitle{}
	\vfill
	\textit{Oxidační stav VI}
	\begin{itemize}
		\item Manganany jsou nestálé a v kyselém prostředí disproporcionují:
		\item \ce{K2MnO4 + 4 HCl -> 2 KMnO4 + MnO2 + 4 KCl + 2 H2O}
		\item Redukcí manganistanu pomocí \ce{SO2} získáme oxid-chlorid manganový, hnědou kapalinu, která ochotně hydrolyzuje.
		\item Manganany lze připravit zahříváním burelu s hydroxidy alkalických kovů:
		\item \ce{2 MnO2 + 4 KOH + O2 -> 2 K2MnO4 + 2 H2O}
		\item Manganan lze oxidovat chlorem na manganistan:
		\item \ce{2 K2MnO4 + Cl2 -> 2 KMnO4 + 2 KCl}
	\end{itemize}
	\vfill
}

\subsection{Sloučeniny manganu v oxidačním stavu V}
\frame{
	\frametitle{}
	\vfill
	\textit{Oxidační stav V}
	\begin{itemize}
		\item Binární halogenidy nejsou známy.
		\item Jediným známým oxid-halogenidem je \ce{MnOCl3}:\footnote[frame]{\href{https://doi.org/10.1016/0022-1902(68)80421-X}{New and unusual compounds of manganese: \ce{MnO3Cl}, \ce{MnO2Cl2} and \ce{MnOCl3} with remarks on \ce{Mn2O7}}}
		\item \ce{KMnO4 + HSO3Cl -> MnOCl3 + KHSO4}
		\item Je hydrolyticky nestabilní, ochotně hydrolyzuje na manganičnany.
		\item Roztoky manganičnanů jsou stabilní jen v silně zásaditém prostředí, jinak dochází k disproporcionaci:
		\item \ce{2 MnO$_4^{3-}$ + 2 H2O -> MnO$_4^{2-}$ + MnO2 + 4 OH-}
	\end{itemize}
	\vfill
}

\subsection{Sloučeniny manganu v oxidačním stavu IV}
\frame{
	\frametitle{}
	\vfill
	\textit{Oxidační stav IV}
	\begin{itemize}
		\item Jediným známým halogenidem je \ce{MnF4}, vzniká přímou reakcí prvků.
		\item Další možnou přípravou je reakce \ce{MnF2} s fluorem při ozařování UV zářením:\footnote[frame]{\href{https://doi.org/10.1016/S0022-1139(01)00566-8}{Room temperature syntheses of \ce{MnF3}, \ce{MnF4} and hexafluoromanganete(IV) salts of alkali cations}}
		\item \ce{MnF2 + F2 ->[HF, UV, RT] MnF4}
		\item Je to modrá, pevná látka, která se samovolně rozkládá:
		\item \ce{2 MnF4 <=> 2 MnF3 + F2}
		\item Oxid manganičitý, \ce{MnO2}, vytváří několik polymorfních forem. Zpravidla jsou nestechiometrické, pouze vysokoteplotní forma $\beta$-\ce{MnO2} je stechiometrická.
		\item Má oxidační vlastnosti:
		\item \ce{MnO2 + 4 HCl ->[T] MnCl2 + Cl2 + 2 H2O}
	\end{itemize}
	\vfill
}

\frame{
	\frametitle{}
	\vfill
	\begin{itemize}
		\item Reakcí manganistanu s fluoridem v prostředí kyseliny fluorovodíkové získáme komplexní hexafluoromanganičitany:\footnote[frame]{\href{https://doi.org/10.1002/ange.19530651108}{Über eine neue Darstellung des Kalium‐hexafluoromanganats(IV)}}
		\item \ce{2 KMnO4 + 2 KF + 10 HF + 3 H2O2 -> 2 K2MnF6 + 8 H2O + 3 O2}
		\item Struktura je podobná sloučeninám typu \ce{AB2X6}, ionty \ce{K+} a \ce{F-} jsou uspořádány střídavě v kubické a hexagonální mřížce, ionty \ce{Mn^{VI}} jsou umístěny v oktaedrických dutinách a tvoří oktaedry \ce{[MnF6]^{2-}}.
	\end{itemize}
	\begin{figure}
		\adjincludegraphics[height=0.32\textwidth]{img/Alpha-MnF4.png}
		\caption*{Krystalová struktura fluoridu manganičitého.\footnote[frame]{Zdroj: \href{https://commons.wikimedia.org/wiki/File:Alpha-MnF4-from-xtal-1987-CM-3D-polyhedra.png}{Ben Mills/Commons}}}
	\end{figure}
	\vfill
}

\subsection{Sloučeniny manganu v oxidačním stavu III}
\frame{
	\frametitle{}
	\vfill
	\textit{Oxidační stav III}
	\begin{itemize}
		\item Jediným známým binárním halogenidem je fluorid manganitý, \ce{MnF3}.
		\item Snadno hydrolyzuje, ale termicky je stabilní.
		\item Připravuje se oxidací manganatých solí fluorem:\footnote[frame]{\href{https://doi.org/10.1016/S0022-1139(01)00566-8}{Room temperature syntheses of \ce{MnF3}, \ce{MnF4} and hexafluoromanganete(IV) salts of alkali cations}}
		\item \ce{2 MnF2 + F2 -> 2 MnF3}
		\item S alkalickými halogenidy vytváří hexafluoromanganitany:
		\item \ce{MnF3 + 3 NaF -> Na3MnF6}
		\item Oktaedrický anion \ce{MnF$_6^{3-}$}, stejně jako další vysokospinové komplexy s~konfigurací d$^4$ je deformován vlivem Jahnova-Tellerova efektu.\footnote[frame]{\href{https://chem.libretexts.org/Bookshelves/Inorganic_Chemistry/Modules_and_Websites_(Inorganic_Chemistry)/Coordination_Chemistry/Structure_and_Nomenclature_of_Coordination_Compounds/Coordination_Numbers_and_Geometry/Jahn-Teller_Distortions}{Jahn-Teller Distortions}}
		\item Naproti tomu, komplex vznikající reakcí \ce{K3MnF6} s KCN je nízkospinový a anionty jsou pravidelné oktaedry:
		\item \ce{K3MnF6 + 6 KCN -> K3[Mn(CN)6] + 6 KF}
	\end{itemize}
	\vfill
}

\frame{
	\frametitle{}
	\vfill
	\begin{itemize}
		\item Oxid manganitý, \ce{Mn2O3}, je černá pevná látka, kterou lze připravit zahříváním \ce{MnO2} na teplotu nad 800~$^\circ$C.
		\item Vzniká také v suchých článcích obsahujících burel:
		\item \ce{2 MnO2 + Zn -> Mn2O3 + ZnO}
		\item Reakcí nadbytku oxidu draselného s oxidem manganitým v zatavené niklové bombičce při teplotě 610~$^\circ$C lze za deset dní připravit červené krystaly dimanganitanu draselného (\ce{K6Mn2O6}).\footnote[frame]{\href{https://doi.org/10.1007/BF00597313}{Das erste Oxomanganat (III) mit Inselstruktur: \ce{K6[Mn2O6]}}}
		\item Tmavěčervený komplexní anion \ce{[Mn(CN)6]^{3-}} můžeme připravit proháněním vzduchu vodným roztokem manganaté soli a kyanidu.
		\item Manganatá sůl je oxidována vzdušným kyslíkem:
		\item \small \ce{4 MnCl2 + O2 + 20 KCN + 4 HCN -> 4 K3[Mn(CN)6] + 2 H2O + 8 KCl}
	\end{itemize}
	\vfill
}

\subsection{Sloučeniny manganu v oxidačním stavu II}
\frame{
	\frametitle{}
	\vfill
	\textit{Oxidační stav II}
	\begin{itemize}
		\item Manganaté soli jsou zpravidla růžové nebo bezbarvé.
		\item Známe všechny halogenidy manganaté.
		\item Chlorid a síran je možné připravit zahříváním burelu s příslušnou kyselinou:
		\item \ce{2 MnO2 + 4 HCl -> 2 MnCl2 + O2 + 2 H2O}
		\item Fluorid a bromid lze připravit reakcí uhličitanu manganatého s příslušnou kyselinou halogenovodíkovou.
		\item \ce{MnCO3 + 2 HF -> MnF2 + CO2 + H2O}
		\item Bezvodé lze získat termickou dehydratací.
		\item Jodid se připravuje přímo z prvků.
		\item Manganaté komplexy jsou zpravidla vysokospinové.
		\item Nízkospinový je pouze kyano komplex \ce{K4[Mn(CN)6]}, který připravíme reakcí vodných roztoků uhličitanu a kyanidu draselného:
		\item \ce{MnCO3 + 6 KCN ->[H2O] K4[Mn(CN)6] + K2CO3}
	\end{itemize}
	\vfill
}

\subsection{Sloučeniny manganu v oxidačním stavu I}
\frame{
	\frametitle{}
	\vfill
	\textit{Oxidační stav I}
	\begin{itemize}
		\item Tento oxidační stav je běžnější u organokovových sloučenin s $\pi$-akceptorními ligandy, např. crownethery nebo pyrazoly.
		\item Reakcí deoxygenovaného vodného roztoku NaCN s práškovým manganem získáme komplex \ce{Na5[Mn(CN)6]}.
		\item Tuto sloučeninu lze připravit i redukcí \ce{Na4[Mn(CN)6]} sodným amalgámem.
		\item Do této skupiny sloučenin patří i trikarbonyl(methylcyklopentadienyl)\\mangan.
	\end{itemize}
	\begin{figure}
	\adjincludegraphics[height=0.35\textwidth]{img/Methylcyclopentadienyl-Manganese-Tricarbonyl_Skeletal.png}
	\end{figure}
	\vfill
}

\subsection{Sloučeniny manganu v oxidačním stavu 0}
\frame{
	\frametitle{}
	\vfill
	\textit{Oxidační stav 0}
	\begin{itemize}
		\item Dekakarbonyldimanganu, \ce{Mn2(CO)5}, je žlutá krystalická látka s teplotou tání 154~$^\circ$C.
		\item Připravuje se redukcí trikarbonyl(methylcyklopentadienyl)manganu sodíkem v atmosféře \ce{N2} a \ce{CO}.\footnote[frame]{\href{https://doi.org/10.1016/0022-328X(68)80099-3}{A convenient synthesis of dimanganese decarbonyl from inexpensive starting materials at atmospheric pressure}}
	\end{itemize}
	\begin{columns}
		\begin{column}{.5\textwidth}
			\begin{figure}
				\adjincludegraphics[height=0.45\textwidth]{img/Mn2-CO-10.png}
			\end{figure}
		\end{column}
		\begin{column}{.5\textwidth}
			\begin{figure}
				\adjincludegraphics[height=0.6\textwidth]{img/Dimanganese-decacarbonyl.png}
				\caption*{Kuličkový model dekakarbonyldimanganu.\footnote[frame]{Zdroj: \href{https://commons.wikimedia.org/wiki/File:Dimanganese-decacarbonyl-3D-balls.png}{Ben Mills/Commons}}}
			\end{figure}
		\end{column}
	\end{columns}
	\vfill
}

\subsection{Sloučeniny manganu v oxidačním stavu $-$I}
\frame{
	\frametitle{}
	\vfill
	\textit{Oxidační stav $-$I}
	\begin{itemize}
		\item Do této skupiny sloučenin patří jeden z nejstabilnějších hydridů první periody přechodných kovů.
		\item Pentakarbonylhydridomangan, \ce{[HMn(CO)5]}, je bezbarvá kapalina.
		\item Lze ho připravit reakcí dekakarbonyldimanganu s borohydridem:
		\item \ce{2 LiHBEt3 + Mn2(CO)5 -> 2 Li[Mn(CO)5] + H2 + 2 Et3B}
		\item \ce{Li[Mn(CO)5] + CF3SO3H -> HMn(CO)5 + Li+CF3SO$_3^-$}
		\item p$K_a$ = 7.1 ve vodném prostředí.\footnote[frame]{\href{https://doi.org/10.1021/acs.chemrev.5b00695}{Brønsted–Lowry Acid Strength of Metal Hydride and Dihydrogen Complexes}}
	\end{itemize}
	\begin{columns}
		\begin{column}{.5\textwidth}
			\begin{figure}
				\adjincludegraphics[height=0.25\textheight]{img/HMn-CO-5.png}
			\end{figure}
		\end{column}
		\begin{column}{.5\textwidth}
		\begin{figure}
			\adjincludegraphics[height=0.25\textheight]{img/Pentacarbonylhydridomanganese.png}
			\caption*{Kuličkový model pentakarbonylhydridomanganu.\footnote[frame]{Zdroj: \href{https://commons.wikimedia.org/wiki/File:Pentacarbonylhydridomanganese-3D-balls.png}{Jynto and Ben Mills/Commons}}}
		\end{figure}
		\end{column}
	\end{columns}
	\vfill
}

\subsection{Sloučeniny TC a Re v oxidačním stavu VII--V}
\frame{
	\frametitle{}
	\vfill
	\textit{Oxidační stavu VII--V}
	\begin{columns}
	\begin{column}{.7\textwidth}
	\begin{itemize}
		\item Kovové rhenium poskytuje s fluorem buď \ce{ReF6} nebo \ce{ReF7}, v závislosti na podmínkách.
		\item \ce{ReF7} je žlutá pevná látka, taje při 48~$^\circ$C.
		\item Je to druhý stabilní heptafluorid, prvním je \ce{IF7}.
		\item Má strukturu deformované pentagonální bipyramidy.\footnote[frame]{\href{https://doi.org/10.1126/science.263.5151.1265}{Crystal and Molecular Structures of Rhenium Heptafluoride}}
		\item Vzniká při teplotě 400 $^\circ$C:
		\item \ce{2 Re + 7 F2 ->[400 $^\circ$C] 2 ReF7}
		\item S fluoridy vytváří komplexní anionty \ce{ReF$_8^-$}.
		\item Reakcí s \ce{SbF5} poskytuje kation \ce{ReF$_6^+$}.
		\item \ce{SbF5 + ReF7 -> [ReF6][SbF6]}
	\end{itemize}
	\end{column}
	\begin{column}{.4\textwidth}
	\begin{figure}
		\adjincludegraphics[width=\textwidth]{img/ReF7.png}
	\end{figure}
	\end{column}
	\end{columns}
	\vfill
}

\frame{
	\frametitle{}
	\vfill
	\textit{Oxidační stavu VII--V}
	\begin{itemize}
		\item \ce{ReF5} se připravuje rozkladem na wolframovém vlákně:
		\item \ce{ReF6 ->[600 $^\circ$C] ReF5}
		\item nebo reakcí s fluoridem fosforitým:\footnote[frame]{\href{https://doi.org/10.1071/CH9710243}{Reactivity of transition metal fluorides. VIII. Reactions of rhenium heptafluoride and hexafluoride}}
		\item \ce{2 ReF6 + PF3 -> 2 ReF5 + PF5}
		\item Přímá reakce Tc s fluorem vede k \ce{TcF6} a \ce{TcF5}.
		\item \ce{TcF7} nebyl zatím připraven.
		\item Reakce s chlorem a bromem vedou k halogenidům v oxidačním stupni VI a V. Chlorid rheničný je dimerní, \ce{Re2Cl10}.
	\end{itemize}
	\begin{figure}
		\adjincludegraphics[height=0.23\textwidth]{img/Re2Cl10.png}
	\end{figure}
	\vfill
}

\frame{
	\frametitle{}
	\vfill
	\begin{itemize}
		\item Těkavé oxidy, \ce{M2O7}, vznikají spalováním kovů v kyslíku. Jsou anhydridy kyselin (obě jsou silné):
		\item \ce{Tc2O7 + H2O -> 2 HTcO4}
		\item Kyseliny \ce{HMO4} byly izolovány i v krystalickém stavu, jejich soli jsou isostrukturní s \ce{MnO$_4^-$}.
		\item Jsou tvořeny tetraedry \ce{MO4} spojenými vrcholy.
		\item Technecistany a rhenistany jsou běžnou výchozí látkou pro studium sloučenin těchto kovů.
		\item Oxid rheniový lze připravit redukcí oxidem uhelnatý, oxid techneciový nebyl zatím připraven:
		\item \ce{Re2O7 + CO -> 2 ReO3 + CO2}
	\end{itemize}
	\begin{figure}
		\adjincludegraphics[width=0.9\textwidth]{img/M2O7.png}
	\end{figure}
	\vfill
}

\subsection{Sloučeniny TC a Re v oxidačním stavu IV}
\frame{
	\frametitle{}
	\vfill
	\begin{columns}
	\begin{column}{.65\textwidth}
		\textit{Oxidační stav IV}
		\begin{itemize}
		\item Modrý \ce{ReF4} můžeme připravit redukcí \ce{ReF5} vodíkem na platinové síťce.
		\item \ce{ReCl4} se připravuje komproporcionací \ce{ReCl5} a \ce{Re3Cl9} za vyšší teploty.
		\item \ce{TcCl4} lze připravit reakcí \ce{Tc2O7} s \ce{CCl4}.
		\item Oxidy \ce{MO2} se připravují redukcí \ce{M2O7} pomocí čistého kovu nebo vodíku.
		\item Známe všechny oktaedrické komplexy typu \ce{MX$^{2-}_6$}.
	\end{itemize}
	\end{column}
	\begin{column}{.4\textwidth}
	\begin{figure}
		\adjincludegraphics[width=\textwidth]{img/Re3Cl9.png}
		\caption*{Struktura \ce{Re3Cl9}}
	\end{figure}
	\end{column}
	\end{columns}
	\vfill
}

\subsection{Sloučeniny TC a Re v oxidačním stavu III}
\frame{
	\frametitle{}
	\vfill
	\textit{Oxidační stav III}
	\begin{itemize}
		\item U sloučenin v oxidačním stavu III je běžná vazba kov-kov, halogenidy jsou trimerní.
		\item Halogenidy ochotně vytvářejí adukty s lewisovými bazemi:
		\item \ce{[Re3Cl9(py)3] <-[py] [Re3Cl9] ->[PR3] [Re3Cl9(PR3)]}
		\item Anion \ce{Re2Cl$_8^{2-}$} byl první sloučeninou, u které byla prokázána \textit{čtverná vazba}.\footnote[frame]{\href{https://doi.org/10.1021/ic50025a015}{The Crystal and Molecular Structure of Dipotassium Octachlorodirhenate(III) Dihydrate, \ce{K2[Re2Cl8].2H2O}}}
	\end{itemize}
	\begin{figure}
		\adjincludegraphics[height=.33\textheight,angle=90]{img/Octachlorodirhenate.png}
		\caption*{Kuličkový model \ce{Re2Cl$_8^{2-}$}.\footnote[frame]{Zdroj: \href{https://commons.wikimedia.org/wiki/File:Octachlorodirhenate(III)-3D-balls.png}{Benjah-bmm27/Commons}}}
	\end{figure}
	\vfill
}

\subsection{Sloučeniny TC a Re v oxidačním stavu I}
\frame{
	\frametitle{}
	\vfill
	\textit{Oxidační stav I}
	\begin{itemize}
		\item Oxidační stav I je stabilizován $\pi$-akceptorními ligandy, např. CO.
		\item Tyto sloučeniny \ce{^{99m}Tc} jsou zajímavé pro diagnostické snímkování, např. \ce{[Tc(H2O)3(CO)3]+}.
		\item Oktaedrický kation \ce{[Tc(RNC)6]+} můžeme připravit redukcí technecistanu pomocí dithioničitanu \ce{S2O$_4^{2-}$} v přítomnosti izokyanidu RNC.
	\end{itemize}
	\begin{figure}
		\adjincludegraphics[height=.4\textheight]{img/Tc-H2O-CO.png}
	\end{figure}
	\vfill
}

\subsection{Sloučeniny Tc a Re v oxidačním stavu 0}
\frame{
	\frametitle{}
	\vfill
	\textit{Oxidační stav 0}
	\begin{itemize}
		\item Dekakarbonyldirhenia, \ce{Rh2(CO)5}, je bílá krystalická látka s teplotou tání 170~$^\circ$C.
		\item Připravuje se redukcí rhenistanu draselného sodíkem v atmosféře  \ce{CO} za vysokého tlaku a teploty.\footnote[frame]{\href{https://doi.org/10.1016/S0022-328X(00)89806-X}{Molecular structure of dirhenium decacarbonyl}}
	\end{itemize}
	\begin{columns}
		\begin{column}{.5\textwidth}
			\begin{figure}
				\adjincludegraphics[height=0.4\textwidth]{img/Re2-CO-10.png}
			\end{figure}
		\end{column}
		\begin{column}{.5\textwidth}
			\begin{figure}
				\adjincludegraphics[height=0.6\textwidth]{img/Rhenium-carbonyl-3D-balls.png}
				\caption*{Kuličkový model dekakarbonyldirhenia.\footnote[frame]{Zdroj: \href{https://commons.wikimedia.org/wiki/File:Rhenium-carbonyl-3D-balls.png}{Ben Mills/Commons}}}
			\end{figure}
		\end{column}
	\end{columns}
	\vfill
}

\subsection{Organokovové sloučeniny}
\frame{
	\frametitle{}
	\begin{columns}
		\begin{column}{.7\textwidth}
			\vfill
			\begin{itemize}
				\item \textit{Methyl-trioxorhenium}
				\item \ce{MeReO3} -- těkavá, bezbarvá pevná látka. Využívá se jako katalyzátor.
				\item Oxidační stav rhenia je +VII.
				\item Lze jej připravit reakcí oxidu rhenistého s tetramethylcínem
				\item \ce{Re2O7 + Me4Sn -> MeReO3 + Me3SnOReO3}.
				\item Katalyzuje mnoho typů reakcí, např.:\footnote[frame]{\href{https://www.organic-chemistry.org/chemicals/oxidations/mto-methyltrioxorhenium.shtm}{Methyltrioxorhenium (MTO)}}
				\begin{itemize}
					\item Metateze olefinů
					\item Oxidaci alkynů na kyseliny nebo diony
					\item Oxidaci alkenů na epoxidy
				\end{itemize}
			\end{itemize}
			\vfill
		\end{column}
		\begin{column}{.35\textwidth}
			\begin{figure}
				\adjincludegraphics[width=\textwidth]{img/Methylrhenium-trioxide-3D-balls.png}
				\caption*{Kuličkový model methyl-trioxorhenia.\footnote[frame]{Zdroj: \href{https://commons.wikimedia.org/wiki/File:Methylrhenium-trioxide-3D-balls.png}{Ben Mills/Commons}}}
			\end{figure}
		\end{column}
	\end{columns}
}

\input{../Last}

\end{document}